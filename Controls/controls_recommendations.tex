%%
%%  Conclusions and Recommendations
%%

\indent\indent In conclusion, this year's iteration of the controls \& instrumentations subsystem introduced major upgrades and capabilities that were lacking in the previous generation. One of the most noticeable upgrades is the introduction of a ground control module that permits real-time monitoring and control of the system. The codebase was rewritten from scratch using a compiler language, particularly C/C++, which gave the subsystem a huge boost in performance when compared to the previous iteration's codebase, which was written in Python, an interpreted language. The codebase is maintained and version tracked using a git repository to ensure future teams have access to the most up-to-date code that was developed by the CS and MAE teams.

The choice of an embedded system to act as the brains of the subsystem proved to be a success as well. The RPi's small footprint, low power requirements, ample GPIO pins, and great community support was more than suitable for the job and future teams are encouraged not to replace them with MCU's due to their ease of use. Lastly, future teams are encouraged to replace the current PCB with a more professional one that uses surface mounted elements instead of through-hole mounting, this will greatly reduce the footprint of the PCB in exchange for maintenance. Nonetheless, the components used and PCB manufacturing are rather inexpensive. However, this is a trade-off that eventually boils down to personal preference.

\subsection*{Controls \& Instrumentations Recommendations}

\indent\indent Nothing is perfect, and this project is no exception. The introduction of a CS team in this year's iteration was a great idea given their educational background and skill set as demonstrated by the ground control module they provided. As with any project, constant communication between both teams had to take place. However, due to the lack of regular weekly face-time between both teams, miscommunication over expectations and deliverables manifested themselves. This has caused delays and on various occasions were the source of awkward, if not unpleasant, interactions. In addition, given that main routines and algorithms for the entire system controls was maintained by the CS team, the MAE team had to wait for the CS team to integrate whatever code was produced into their codebase for integrated testing, and more often than not those requests were looked over, and whenever they were not, it took multiple weeks, and sometimes even months, for the CS team to respond and complete the MAE requests for integration. This effectively hindered  the MAE team's ability to properly integrate and test the controls \& instrumentations subsystems with the mechanical hardware that was available and ready for testing.

Future teams are recommended to stay on top of their out-of-department colleagues in case the next iteration of the project is going to be interdisciplinary as was the case for this year's iteration to avoid any miscommunication and to ensure deliverables are met in a timely manner. 